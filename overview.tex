
Курс ``Функциональное программирование'' посвящен изучению функционального 
стиля программирования. За последнее десятилетие функциональное 
программирование из предмета академических исследований превратилось в
инструмент, позволяющий эффективно решать промышленные задачи. Наблюдается
взрывной рост популярности языков Erlang, Scala, F#. Функциональные 
идиомы являются критической частью современных поисковых машин,
высоконагруженных систем и систем массированной обработки данных. Целый
ряд приемов и языковых конструкций, которые используются в современных
императивных языках программирования, заимствованы из функционального мира.


Цели курса:

- Дать студентам представление о теоретических основах и практической
реализации современных функциональных языков программирования.

- Познакомить студентов с методом функциональной декомпозиции задачи
и с характерными для функционального стиля приемами программирования.

- Рассмотреть наиболее важные приёмы из мира функционального программирования:
рекурсивные и итеративные процессы, функции высших порядков и замыкания,
абстрактные типы данных, свёртки, доменно-специфичные мини-языки, 
модель окружений и бесконечные потоки данных а также дать введение в
системы типов.

В результате изучения курса студенты приобретают практические навыки
решения задач с помощью функциональных языков программирования.


Литература курса:

- Абельсон Х., Сассман Дж. ``Структура и интерпретация компьютерных программ''
- Харрисон Дж. ``Введение в функциональное программирование''
- Харрисон П., Филд А. ``Функциональное программирование''
- Симон П.-Д., ``Язык и библиотеки Haskell 98''
- Пирс Б., ``Типы в языках программирования''

