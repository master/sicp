\documentclass[a4paper,11pt]{article}
\usepackage[utf8]{inputenc}
\usepackage[russian]{babel}

\author{Олег Смирнов \\
\texttt{oleg.smirnov@gmail.com}}
\date{\today}
\title{Функциональное программирование -- аннотация курса}

\begin{document}

\maketitle

Курс ``Функциональное программирование'' посвящён изучению функционального 
стиля программирования. За последнее десятилетие функциональное 
программирование из предмета академических исследований превратилось в
инструмент, позволяющий эффективно решать промышленные задачи. Наблюдается
взрывной рост популярности языков Erlang, Scala, F\#. Функциональные 
идиомы являются критической частью современных поисковых машин,
высоконагруженных систем и систем массированной обработки данных. Целый
ряд приёмов и языковых конструкций, которые используются в современных
императивных языках программирования, заимствованы из функционального мира.

\subsection*{Цели курса}
\begin{itemize}
\item Дать студентам представление о теоретических основах и практической
реализации современных функциональных языков программирования
\item Познакомить студентов с методом функциональной декомпозиции задачи
и с характерными для функционального стиля приёмами программирования.
\item Рассмотреть наиболее важные приёмы из мира функционального
программирования: рекурсивные и итеративные процессы, функции высших порядков
и замыкания, абстрактные типы данных, доменно-специфичные языки, модель
окружений и бесконечные потоки данных, а также дать введение в системы типов
\end{itemize}

В результате изучения курса студенты приобретают практические навыки
решения задач с помощью функциональных языков программирования.

\subsection*{Литература курса}
\begin{itemize}
\item Абельсон Х., Сассман Дж. ``Структура и интерпретация компьютерных программ''
\item Харрисон Дж. ``Введение в функциональное программирование''
\item Харрисон П., Филд А. ``Функциональное программирование''
\item Пейтон-Джонс С. ``Язык и библиотеки Haskell 98''
\item Пирс Б. ``Типы в языках программирования''
\end{itemize}
\end{document}
