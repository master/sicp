\documentclass[a4paper,11pt]{article}
\usepackage[utf8]{inputenc}
\usepackage[russian]{babel}
\usepackage{hyperref}

\author{Олег Смирнов \\
\texttt{oleg.smirnov@gmail.com}}
\date{\today}
\title{Функциональное программирование -- программа курса}

\begin{document}

\maketitle

\begin{abstract}
Программа курса базируется на курсе ``CS61A'' университета Беркли\footnote{
\href{http://goo.gl/kl7tc}{http://wla.berkeley.edu/~cs61a/sp11/}}, на курсе
``6.001: Structure and Interpretation of Computer Programs'' МТИ\footnote{
\href{http://goo.gl/MW2ME}
{http://ocw.mit.edu/ ... /6-001-structure-and-interpretation-of-computer-programs}},
а также на ряде курсов, читавшихся на факультете технической кибернетики СПбПГУ,
факультете кибернетики МИФИ\footnote{\href{http://goo.gl/pbV08}
{http://ru.wikibooks.org/wiki/Основы\_функционального\_программирования}},
в CS-клубе ПОМИ\footnote{
\href{http://goo.gl/N0la1}{http://logic.pdmi.ras.ru/csclub/courses/fprog}}
и в ряде других ВУЗов.
\end{abstract}

\section*{Раздел 1. Основы функционального программирования}

\subsection*{Тема 1.1. Обзор курса}
История функционального программирования. Сравнение императивного и
функционального подхода. Свойства функциональных языков.

\subsection*{Тема 1.2. Основы функционального программирования}
Язык Scheme. Рекурсия и хвостовые вызовы. Функции как абстракции. 
Нормальный и аппликативный порядок вычислений. Пример: вычисление 
квадратного корня методом Ньютона. 
\\\\
Литература: Абельсон \& Сассман, глава 1.1

\subsection*{Тема 1.3. Функции высших порядков. Замыкания}
Функции как аргументы и как возвращаемые значения. Особенности функций
высших порядков в языках без сборки мусора.
\\\\
Литература: Абельсон \& Сассман, глава 1.3

\subsection*{Тема 1.4. Рекурсивные и итеративные процессы}
Линейная рекурсия и итерация. Древовидная рекурсия. Эффективность алгоритмов.
Пример: эффективный алгоритм возведения в степень.
\\\\
Литература: Абельсон \& Сассман, главы 1.2-1.2.4

\subsection*{Тема 1.5. Абстракция данных}
Барьеры абстракции. Последовательности. Составной тип данных. Конструкторы,
селекторы и предикаты. Пример: MapReduce.
\\\\
Литература: Абельсон \& Сассман, главы 2.1 и 2.2.1

\subsection*{Тема 1.6. Иерархические структуры}
Деревья и рекурсивный обход. Оператор кавычки. Интерпретатор языка Scheme.
\\\\
Литература: Абельсон \& Сассман, главы 2.2.2-2.2.3, 2.3.1 и 2.3.3

\subsection*{Тема 1.7. Обобщённые операторы}
Метки для данных. Обобщённые операции и основы полиморфизма. Обмен
сообщениями. Пример: обобщённые арифметические операции.
\\\\
Литература: Абельсон \& Сассман, главы 2.4-2.5.2

\section*{Раздел 2. Практика функционального программирования}

\subsection*{Тема 2.1. Объектно-ориентированное программирование}
Основы ООП в функциональном программировании: обмен сообщениями,
внутреннее состояние, наследование.
\\\\
Литература: Абельсон \& Сассман, глава 3.1

\subsection*{Тема 2.2. Модель вычислений с окружениями}
Внутреннее состояние и глобальное окружение. Правила вычисления. Кадры как
способ организации внутреннего состояния.
\\\\
Литература: Абельсон \& Сассман, главы 3.1 и 3.2

\subsection*{Тема 2.3 Изменяемое состояние}
Понятие изменяемого состояния. Списковая структура. Представление очередей
и таблиц.
\\\\
Литература: Абельсон \& Сассман, главы 3.3.1-3.3.3

\subsection*{Тема 2.4 Конкурентные вычисления}
Параллелизм и конкурентные вычисления. Механизмы управления конкурентными
процессами. Синхронизация с помощью мьютексов. Пример: клиент-серверное
приложение.
\\\\
Литература: Абельсон \& Сассман, глава 3.4

\subsection*{Тема 2.5. Потоки}
Парадигма потоков. Потоки как задержанные списки. Бесконечные потоки данных.
Задержанные вычисления. Синхронизация с помощью ``фьючеров'' и ``обещаний''.
\\\\
Литература: Абельсон \& Сассман, главы 3.5.1-3.5.3, 3.5.5

\subsection*{Тема 2.6. Метациклический интерпретатор}
Универсальность языка. Выполнение интерпретатора как программы. Пара
apply/eval. Пример: MapReduce, часть 2.
\\\\
Литература: Абельсон \& Сассман, глава 4.1

\subsection*{Тема 2.7. Метациклический интерпретатор, часть 2. Доменно-специфичные языки}
Формулировка логики программы в терминах предметной области. Синтаксический
анализ и правила выполнения. Пример: система переписывания термов.
\\\\
Литература: Абельсон \& Сассман, глава 4.2

\section*{Раздел 3. Введение в системы типов}

\subsection*{Тема 3.1. Системы типов (начало)}
Типизированное и нетипизированное лямбда-исчисление. Понятия систем типов.
Язык Haskell. Обобщённые алгебраические типы.
\\\\
Литература: Симон, главы 2-5. Пирс, главы 2-4

\subsection*{Тема 3.2. Универсальный полиморфизм и вывод типов}
Параметрический полиморфизм. Классы типов в языке Haskell. Модель типизации
Хиндли—Милнера. Пример: реализация алгоритма Хиндли-Милнера.
\\\\
Литература: Симон, глава 6. Пирс, глава 5

\end{document}
