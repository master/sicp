Программа курса базируется на курсе ``6.001: Structure and Interpretation of
Computer Programs'' МТИ, на курсе CS61A университета Беркли, а также на ряде
курсов с аналогичной программой, которые читались на факультете технической
кибернетики СПбПГУ, факультете кибернетики МИФИ, в CS-клубе ПОМИ и в ряде
других ВУЗов.


Раздел 1. Основы функционального программирования

Раздел 1. Тема 1.1. Обзор курса. История функционального программирования.
Сравнение императивного и функционального подхода. Свойства функциональных
языков.


Раздел 1. Тема 1.2. Основы функционального программирования.
Язык Scheme. Рекурсия и хвостовые вызовы. Функции как абстракции. 
Нормальный и аппликативный порядок вычислений. Пример: вычисление 
квадратного корня методом Ньютона. 

Литература: Абельсон & Сассман, глава 1.1


Раздел 1. Тема 1.3. Функции высших порядков. Замыкания.
Функции как аргументы и как возвращаемые значения. Особенности функций
высших порядков в языках без сборки мусора.

Литература: Абельсон & Сассман, глава 1.3


Раздел 1. Тема 1.4. Рекурсивные и итеративные процессы.
Линейные рекурсия и итерация. Древовидная рекурсия. Эффективность алгоритмов.
Пример: эффективный алгоритм возведения в степень.

Литература: Абельсон & Сассман, главы 1.2 - 1.2.4


Раздел 1. Тема 1.5. Абстракция данных.
Барьеры абстракции. Последовательности. Составной тип данных. Конструкторы,
селекторы и предикаты. Пример: MapReduce.

Литература: Абельсон & Сассман, главы 2.1 и 2.2.1


Раздел 1. Тема 1.6. Иерархические структуры.
Деревья и рекурсивный обход. Оператор кавычки. Интерпретатор языка Scheme.

Литература: Абельсон & Сассман, главы 2.2.2-2.2.3, 2.3.1 и 2.3.3


Раздел 1. Тема 1.7. Обобщённые операторы.
Метки для данных. Обобщенные операции и основы полиморфизма. Обмен сообщениями.
Пример: обобщённые арифметические операции.

Литература: Абельсон & Сассман, главы 2.4 - 2.5.2


Раздел 2. Практика функционального программирования

Раздел 2. Тема 2.1. Объектно-ориентированное программирование.
Основы ООП: обмен сообщениями, внутреннее состояние, наследование.

Литература: Абельсон & Сассман, глава 3.1


Раздел 2. Тема 2.2. Модель вычислений с окружениями.
Внутреннее состояние и глобальное окружение. Правила вычисления. Кадры.

Литература: Абельсон & Сассман, главы 3.1 и 3.2


Раздел 2. Тема 2.3 Изменяемое состояние.
Понятие изменяемого состояния. Списковая структура. Представление очередей
и таблиц.

Литература: Абельсон & Сассман, главы 3.3.1 - 3.3.3


Раздел 2. Тема 2.4 Конкурентные вычисления.
Параллелизм и конкурентные вычисления. Механизмы управления конкурентными
задачами. Синхронизация с помощью мьютексов. Пример: клиент-серверное приложение.

Литература: Абельсон & Сассман, глава 3.4


Раздел 2. Тема 2.5. Потоки.
Парадигма потоков. Потоки как задержанные списки. Бесконечные потоки данных.
Задержанные вычисления. Синхронизация с помощью ``фьючеров'' и ``обещаний''.

Литература: Абельсон & Сассман, главы 3.5.1-3.5.3, 3.5.5


Раздел 2. Тема 2.6. Метациклический интерпретатор.
Универсальность языка. Выполнение интерпретатора как программы. Пара apply/eval.
Пример: MapReduce, часть 2.

Литература: Абельсон & Сассман, глава 4.1


Раздел 2. Тема 2.7. Метациклический интерпретатор, часть 2. Доменно-специфичные языки.
Формулировка логики программы в терминах предметной области. Синтаксический
анализ и правила выполнения. Пример: система переписывания термов.

Литература: Абельсон & Сассман, глава 4.2


Раздел 3. Введение в системы типов.

Раздел 3. Тема 3.1. Системы типов (начало).
Типизированное и нетипизированное лямбда-исчисление. Понятия систем типов. Язык Haskell. 
Обобщенные алгебраические типы.

Литература: Симон, главы 2-5


Раздел 3. Тема 3.2. Универсальный полиморфизм и вывод типов.
Параметрический полиморфизм. Классы типов в языке Haskell. Модель типизации Хиндли—Милнера.
Пример: реализация алгоритма Хиндли-Милнера.

Литература: Симон, глава 6
